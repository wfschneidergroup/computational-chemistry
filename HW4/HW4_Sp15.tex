% Created 2015-02-24 Tue 13:39
\documentclass[11pt]{article}
\usepackage[utf8]{inputenc}
\usepackage{lmodern}
\usepackage[T1]{fontenc}
\usepackage{fixltx2e}
\usepackage{graphicx}
\usepackage{longtable}
\usepackage{float}
\usepackage{wrapfig}
\usepackage{rotating}
\usepackage[normalem]{ulem}
\usepackage{amsmath}
\usepackage{textcomp}
\usepackage{marvosym}
\usepackage{wasysym}
\usepackage{amssymb}
\usepackage{amsmath}
\usepackage[version=3]{mhchem}
\usepackage[numbers,super,sort&compress]{natbib}
\usepackage{natmove}
\usepackage{url}
\usepackage{minted}
\usepackage{underscore}
\usepackage[linktocpage,pdfstartview=FitH,colorlinks,
linkcolor=blue,anchorcolor=blue,
citecolor=blue,filecolor=blue,menucolor=blue,urlcolor=blue]{hyperref}
\usepackage{attachfile}
\usepackage[left=1in, right=1in, top=1in, bottom=1in, nohead]{geometry}
\usepackage{fancyhdr}
\usepackage{hyperref}
\usepackage{setspace}
\usepackage[labelfont=bf]{caption}
\usepackage{amsmath}
\usepackage{enumerate}
\usepackage[parfill]{parskip}
\date{Due: 03/02/2015}
\title{}
\begin{document}

\title{Homework 4\\Lectures 5: Potential Energy Sufaces\\(CBE 60553)}
\author{Prof. William F.\ Schneider}
\maketitle


\section{Isomerization Reaction of Acetaldehyde}
\label{sec-1}

Your task is to calculate properties of the isomerization reaction of acetaldehyde going to vinyl alcohol at the HF/6-31G(d) level:

\begin{center}
\includegraphics[width=.9\linewidth]{fig1.png}
\end{center}


\begin{enumerate}[(a)]
\item First, optimize the structures of the reactant acetaldehyde and product vinyl alcohol. Make a table of the key internal coordinates in the two, including the H1–C, C–C, C–O, and O–H1 distances.

\item Which geometry optimization method did you use in (a)? What type of coordinate system? What were the convergence criteria?

\item Calculate the vibrational spectra of the reactant and product. Confirm that both are true minima (if not, adjust and recalculate). Identify the most prominent (intense) infrared vibrational modes in the two. Could you distinguish these two by their vibrational spectra?

\item Do a rigid scan about the H1-C-C-O dihedral in acetaldehyde. It is easiest to do this using a z-matrix representation of acetaldehyde. Construct a z-matrix based on your optimized acetaldehyde structure and do a series of energy evaluations as you vary the dihedral angle. Approximately how large is the barrier to rotation about the C–C bond?
\end{enumerate}


Recall the z-matrix looks like:

\begin{verbatim}
C1
C2 1 dCC
O 1 dCO 2 aOCC
H1 2 dCH1 1 aHCC1 3 dHCCO 

...
\end{verbatim}


\section{Calculating Transition States}
\label{sec-2}

Now calculate the transition state between acetaldehyde and vinyl alcohol.


\begin{itemize}
\item Start by guessing a structure near the transition state (note I gave you some hints in the figure), and calculating the Hessian to make sure you are near a saddle point.

\item Search from this starting point for the transition state.

\item Make sure your calculation converges, and calculate the vibrational spectrum again to make sure you landed at the saddle point. Note the imaginary vibrational mode and add the key internal coordinates to the table in 1(a).

\item Perform single-point MP2/6-31G(d) calculations on the reactant, transition state, and product. Use the results to complete the table of relative energies below:
\end{itemize}

\begin{center}
\begin{tabular}{lll}
TS & Reactant  (kJ/mol) & Product-reactant  (kJ/mol)\\
\hline
HF/6-31G(d) &  & \\
MP2/6-31G(d) &  & \\
ZPE &  & \\
MP2 + ZPE &  & \\
\end{tabular}
\end{center}


\section{Effect of substituents}
\label{sec-3}

Your colleague wants to know if replacing one of the methyl H’s with an F will speed-up or slow down the isomerization. You know from experience that it is much easier to calculate relative rates than absolute ones. Perform additional calculations to determine whether the reaction barrier is higher or lower with the F substituent.

\section{Electronic Transitions}
\label{sec-4}

Impressed by your ability to predict things, your colleague now wants to know which of acetaldehyde and vinyl alcohol has the lower energy first electronic transition. Perform a CIS/6- 31G(d) calculation on each to estimate the energy of the first excited state and the wavelength of light needed to excite the molecule to that state.

\section{Useful Templates}
\label{sec-5}

\subsection{Frequency calculation:}
\label{sec-5-1}
\begin{verbatim}
$CONTRL SCFTYP=RHF RUNTYP=HESSIAN $END
$BASIS GBASIS=N31 NGAUSS=6 NDFUNC=1 $END
$FORCE METHOD=ANALYTIC VIBANL=.TRUE. $END
$GUESS GUESS=MOREAD NORB=xxx $END ! use if you have a converged SCF wavefunction to read in 
$DATA
... 
$END
\end{verbatim}


\subsection{Geometry optimization using redundant internal coordinates:}
\label{sec-5-2}
\begin{verbatim}
$CONTRL SCFTYP=RHF RUNTYP=OPTIMIZE NZVAR=”3n-6” $END 
$BASIS GBASIS=N31 NGAUSS=6 NDFUNC=1 $END
$STATPT NSTEP=xx $END
$ZMAT DLC=.TRUE. AUTO=.TRUE. $END
$GUESS GUESS=MOREAD NORB=xxx $END ! use if you have a converged SCF wavefunction to read in
$DATA
 ...
$END
$VEC ! converged SCF wavefunction, if you have it 
...
$END
\end{verbatim}


\subsection{Transition state search:}
\label{sec-5-3}
\begin{verbatim}
$CONTRL SCFTYP=RHF RUNTYP=SADPOINT NZVAR=”3n-6” $END 
$BASIS GBASIS=N31 NGAUSS=6 NDFUNC=1 $END
$STATPT HESS=READ NSTEP=xx $END
$ZMAT DLC=.TRUE. AUTO=.TRUE. $END
$GUESS GUESS=MOREAD NORB=xxx $END ! use if you have a converged SCF wavefunction to read in
$DATA
... 
$END 
$HESS
... 
$END 
$VEC
... 
$END
\end{verbatim}


\subsection{MP2 calculation:}
\label{sec-5-4}

\begin{verbatim}
$CONTRL SCFTYP=RHF RUNTYP=ENERGY MPLEVL=2 $END 
$BASIS GBASIS=N31 NGAUSS=6 NDFUNC=1 $END
$DATA !
...
$END
\end{verbatim}


\subsection{CIS calculation:}
\label{sec-5-5}

\begin{verbatim}
$CONTRL SCFTYP=RHF RUNTYP=ENERGY CITYP=CIS $END 
$BASIS GBASIS=N31 NGAUSS=6 NDFUNC=1 $END
$DATA !
...
$END
\end{verbatim}
% Emacs 24.4.1 (Org mode 8.2.10)
\end{document}